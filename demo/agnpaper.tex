%% LyX 2.1.4 created this file.  For more info, see http://www.lyx.org/.
%% Do not edit unless you really know what you are doing.
\documentclass[a4paper,british,a4paper,fleqn,usenatbib,unicode=true]{mnras}
\usepackage[T1]{fontenc}
\usepackage[latin9]{inputenc}
\setcounter{secnumdepth}{3}
\setcounter{tocdepth}{3}
\usepackage{babel}
\usepackage{array}
\usepackage{textcomp}
\usepackage{amsmath}
\usepackage{graphicx}
\usepackage{wasysym}
\usepackage{esint}
\usepackage[authoryear]{natbib}
\usepackage[unicode=true]
 {hyperref}

\makeatletter

%%%%%%%%%%%%%%%%%%%%%%%%%%%%%% LyX specific LaTeX commands.
\pdfpageheight\paperheight
\pdfpagewidth\paperwidth

%% Because html converters don't know tabularnewline
\providecommand{\tabularnewline}{\\}

%%%%%%%%%%%%%%%%%%%%%%%%%%%%%% User specified LaTeX commands.
\input{../allresults/sphereresults.all/stats.tex}
\newcommand{\NH}{ {N_{\rm{H}}} }
\newcommand{\LX}{ {L_{\rm{X}}} }
\newcommand{\AP}{ {p} }

\makeatother

\begin{document}
\title[Galaxy-scale obscuration in Active Galactic Nuclei]{Galaxy
gas as obscurer: II. Separating the galaxy-scale and nuclear obscurers
of Active Galactic Nuclei }
\author[Buchner et al.]{
Johannes Buchner, $^{1,2}$\thanks{E-mail: johannes.buchner.acad@gmx.com}
Franz Bauer$^{1,2}$, 
[EAGLE collaborators]
\\
$^{1}$Millenium Institute of Astrophysics, Vicu\~{n}a. MacKenna 4860, 7820436 Macul, Santiago, Chile
\\
$^{2}$Pontificia Universidad Cat�lica de Chile, Instituto de Astrof�sica, Casilla 306, Santiago 22, Chile
}{


\date{Accepted XXX. Received YYY; in original form ZZZ}
\pubyear{2015}
\label{firstpage}
\pagerange{\pageref{firstpage}--\pageref{lastpage}}

\maketitle
\begin{abstract}The ``torus'' obscurer of Active Galactic Nuclei,
is poorly understood in terms of its density, substructure and physical
mechanisms. Large X-ray surveys provide model boundary constraints,
for both Compton-thin and Compton-thick levels of obscuration, as
obscured fractions are mean covering factors $f_{\text{cov}}$. However,
a major remaining uncertainty is host galaxy obscuration. In Paper
I we discovered a relation of $\NH\propto M_{\star}^{1/3}$ for the
obscuration of galaxy-scale gas. Here we apply this observational
relation to the AGN population, and find that galaxy-scale gas is
responsible for a luminosity-independent fraction of Compton-thin
AGN, but does not produce Compton-thick columns. With the host galaxy
obscuration understood, we present a model of the remaining, nuclear
obscurer which is consistent with a range of observations. The PuffedTorus
model consists of a Compton-thick component ($f_{\text{cov}}\sim35\%$)
and a Compton-thin component ($f_{\text{cov}}\sim40\%$), which is
present depending on black hole mass and luminosity. This is a useful
summary of observational constraints for torus modellers who would
like to reproduce this behaviour. It can also be employed as a sub-grid
recipe in cosmological simulations which do not resolve the torus.
We also investigate host-galaxy X-ray obscuration inside cosmological,
hydro-dynamic simulations (EAGLE, Illustris). The obscuration from
ray-traced galaxy gas can be in agreement with observations, but is
highly sensitive to the chosen feedback assumptions.\end{abstract}

%\begin{keywords}
%galaxies: active -- galaxies: statistics -- X-rays: galaxies -- galaxies: structure -- galaxies: ISM
%\end{keywords}


\section{Introduction}

The vast majority of Active Galactic Nuclei (AGN) are obscured by
thick columns of gas and dust. X-ray surveys over the last decade
indicate that $20-40\%$ are hidden behind Compton-thick column densities
($\NH\apprge10^{24}\text{cm}^{-2}$) and of the remaining population,
$\sim75\%$ are obscured with $\NH=10^{22}-10^{24}\text{cm}^{-2}$
\citep[e.g.][]{Treister2004,Brightman2014,Ueda2014,Buchner2015,Aird2015}
at the peak of AGN activity at redshift $z=0.5-3$. An open question
is whether the same gas reservoir is responsible for fuelling the
AGN by accretion onto Supermassive Black Holes (SMBHs), and whether
it itself is affected by AGN activity. To address this, the first
step is to identify the scale at which the obscuring gas resides.
Traditionally, AGN obscuration is associated with the ``torus'',
a nuclear ($\sim10\text{pc}$) structure around the accretion disk.
Many basic questions about this gas reservoir remain to be answered,
including its density, substructure and stability mechanism \citep{Elitzur2006a,Hoenig2013}.
Assuming sampling from random viewing angles, the large fraction of
obscured AGN implies large covering fractions. Therefore turbulent
structures such as winds from accretion disks have been invoked. However,
for the covering fractions to be useful constraints for torus models,
the importance of galaxy-scale gas to the obscuration has to be estimated.
Separating the covering and column densities from nuclear and galaxy-scale
obscurers is the goal of this work.

Local galaxies exemplify that several scales can contribute to the
obscured columns. The Milky Way gas distribution shows column densities
of $\NH>10^{22}\text{cm}^{-2}$, but only in very low Galactic latitudes
($|b|\apprle2\text{\textdegree}$, \citealp{Dickey1990,Kalberla2005GalNHdist}).
Towards the Galactic Center columns with $\NH>10^{24}\text{cm}^{-2}$
can be found in the Central Molecular Zone \citep{Morris1996,Molinari2011},
as well as in the equivalent central zones of nearby AGN host galaxies
\citep{Prieto2014}. Also the obscuration in the AGN host galaxy NGC~1068
is clearly nuclear (in this work: $\sim100\text{pc}$ or smaller),
because its Compton-thick column is observed in a face-on galaxy \citep{Matt1997a}.
On the other hand, many nearby, obscured AGN are hosted in edge-on
galaxies \citep{Maiolino1995}, which suggests that dust-lanes may
also be important obscurers \citep[see also][for galactic optical/infrared extinction]{Goulding2009}.
Hence \citet{Matt2000} proposed a two-phase model for the obscuration
of AGN: a central, nuclear obscurer which provides Compton-thick obscuration,
and the host galaxy, which provides mildly obscured lines of sight.

However, the Milky Way and local galaxies are limited in their use
as templates for the high-redshift universe, specifically at peak
SMBH growth ($z=0.5-3$; e.g., \citealp{Aird2010}). At that time,
the gas content in galaxies was probably higher, as indicated by molecular
gas measurements \citep[e.g.][]{Tacconi2013} which perhaps lead to
the peak of star formation \citep[see review by][]{Madau2014} and
the increase in the fraction of obscured AGN with redshift \citep[e.g.][]{Treister2004,Ueda2014,Buchner2015}.
The efficient growth of SMBHs at these early epochs has been attributed
to galaxy mergers \citep{Ciotti_Ostriker2001}. This model was expanded
by \citet{Hopkins2006OriginModel} to reproduce local scaling relationships
between galaxy components and SMBH masses, the luminosity function
of AGN, the fraction of active galaxies, and the obscuration dichotomy
of AGN. The evolutionary model also explains why bright AGN are less
frequently found to be mildly obscured than faint AGN \citep{Ueda2003,Silverman2008,Ebrero2009,Ueda2014,Buchner2015,Aird2015},
albeit only qualitatively \citep{Hopkins2005,Hopkins2006OriginModel}.
In this work we investigate the obscuring role of galaxy-scale gas
in the transition from obscured AGN in gas-rich galaxies to unobscured
AGN in gas-poor galaxies, as proposed in the evolutionary sequence
of that model. To effectively decouple the galaxy-scale and nuclear
X-ray obscurer, we need to go beyond a single, central source.

This paper is organised as follows: In Section \ref{sec:Methodology}
we present our computation of the galaxy-scale obscuration using observational
results from Paper I of the obscuring column distribution of galaxies,
applied to the AGN population. Section \ref{sec:Results} presents
our results, which we discuss in Section \ref{sec:Discussion}. With
the galaxy-scale obscurer subtracted, we present a model for the remaining
nuclear obscurer in Section~\ref{sub:PuffedTorus}. 

Independently, Section \ref{sec:Cosmological-simulations} looks into
simulated galaxies in hydro-dynamic cosmological simulations. These
provide predictions for the amount of gas inside galaxies, from which
we derive obscured fractions using ray-tracing. Various model uncertainties
are discussed. Finally we summarise our conclusions in Section~\ref{sec:Summary}.


\section{Methodology}

\label{sec:Methodology}Our goal is to predict the fraction of obscured
AGN from the obscuration of host galaxy-scale gas alone, i.e. without
nuclear obscuration (the torus and central molecular zones). In this
fashion we will be able to separate the large-scale and small-scale
obscurer. In Paper I we established a relation between the distribution
of X-ray absorbing column densities, $\NH$, in galaxies and the stellar
mass of the galaxy. It follows approximately a normal distribution
around 
\begin{equation}
\NH=10^{21.7}\text{cm}^{-2}\times\left(M_{\star}/10^{9.5}M_{\odot}\right)^{1/3}\label{eq:NHM-rel}
\end{equation}
with standard deviation $\sigma=0.5\,\text{dex}$. This $\NH-M_{\star}$
relation was derived using an unbiased sample of long Gamma Ray Bursts
(LGRBs). Since the obscuration is host mass dependent, the obscurer
is arguably the host galaxy itself. In Paper I we show that modern
cosmological hydro-dynamic simulations reproduce absorption by galaxy-scale
metal gas and predict $\NH-M_{\star}$ relations very similar to Equation~\ref{eq:NHM-rel}. 

We apply this relation to the AGN host galaxy population to estimate
host galaxy obscuration. This relation was derived from actively star
forming galaxies, therefore a caveat is that results may slightly
over-represent the galaxy gas present in AGN host galaxies which have
average \citep{Rosario2011,Rosario2013,Santini2012} or even below-average
\citep{Mullaney2015} star formation rates. The major benefit of using
this relationship is that it is based on the same observable as AGN
obscured fractions, namely the photo-electric absorption of X-rays.
Paper I also investigated the host galaxy metallicity bias of LGRB
and concluded that it has a negligible effect on the obscurer. Local
galaxies are also shown there to follow the $\NH-M_{\star}$ relation. 

We start with the stellar mass function (SMF) of the galaxy population.
Its shape $P(M_{\star}|z)$ is approximately a Schechter function
and was measured by e.g. \citet{Muzzin2013} and \citet{Ilbert2013}
out to $z\sim4$. Then we populate the galaxies with AGN. The occupation
probability $P(\text{AGN}|M_{\star},z)$ has been measured by \citet{Aird2012}
and \citet{Bongiorno2012} primarily for the redshift interval $z=0.5-2$.
More accurately, these authors measure the specific accretion rate
distribution (SARD), $P(\LX|M_{\star},z)$, where $\LX$ is the $2-10\,\text{keV}$
X-ray luminosity. They find factorised powerlaw relationships of the
form $P(\LX|M_{\star},z)=A\cdot L_{{\rm X}}^{\gamma_{L}}\cdot M_{\star}^{\gamma_{M}}\cdot(1+z)^{\gamma_{z}}$.
At the highest luminosities, an Eddington limit is required to explain
the steep decrease of the luminosity function \citep{Aird2013}. In
this work, however, we focus on the luminosity range $\LX=10^{42-45}\text{erg/s}$
regime and thus use only the observed relation. The final ingredient
is the obscuring column density distribution (CDD) $P(\NH|M_{\star})$,
which is given by Equation \ref{eq:NHM-rel} as a normal distribution
in $\log\NH$. We assume that the galaxy-scale gas is independent
of nuclear activity for individual galaxies.

The obscured fraction can then be simply computed by Monte Carlo simulations.
Analytically we can put the three distributions together as

\begin{equation}
P(\NH,\LX,M_{\star}|z)=\underbrace{P(M_{\star}|z)}_{\text{SMF}}\times\underbrace{P(\LX|M_{\star},z)}_{\text{SARD}}\times\underbrace{P(\NH|M_{\star})}_{\text{Eq 1}}\label{eq:combination}
\end{equation}
After inserting the factorised powerlaw relationship of the SARD,
the result has the form

\begin{equation}
P(\NH,\LX,M_{\star}|z)=A\times L_{X}^{\gamma_{L}}\times f(M_{\star}|z)\times P(\NH|M_{\star}).
\end{equation}
While the absolute probability of finding an AGN is a function of
luminosity, the mass distribution is independent of luminosity. That
is, at every luminosity (not considering the Eddington limit), the
same mix of host stellar masses contribute. Therefore, the implication
is that the obscuration due to host galaxy-scale gas is the same at
all AGN luminosities.

We adopt an AGN definition of $\LX>10^{42}\text{erg/s}$. The frequency
distribution of column densities $\NH$ for the AGN population is
then computed by integrating over stellar mass and luminosity:

\begin{equation}
P(\NH|z)=\int\int_{10^{42}\text{erg/s}}^{\infty}P(\NH,\LX,M_{\star}|z)\,d\LX\,dM_{\star}.
\end{equation}


Finally the obscured AGN fraction is the cumulative distribution,
i.e. the frequency of AGN being covered by a certain column density
threshold $\NH$ or higher: 
\begin{equation}
f_{\text{cov}}(>\NH)=\int_{\NH}^{\infty}P(\NH'|z)d\NH'.\label{eq:fcov}
\end{equation}
Into the calculation of $f_{\text{cov}}$ we propagate the uncertainties
from the obscuration relation for Paper~I. We consider two SMF (\citealp{Muzzin2013}
and \citealp{Ilbert2013}) and two SARD measurements (\citealp{Aird2012}
and \citealp{Bongiorno2012}), to incorporate systematic uncertainties.
To summarise, we rely only on observational relations to predict the
obscuration of the AGN population by galaxy-scale gas.


\section{Results}

\label{sec:Results}
\begin{figure*}
\begin{centering}
\includegraphics[width=1\textwidth]{/mnt/data/daten/PostDoc/research/agn/sim/compiled/bh/gasobscprofilez_obs}
\par\end{centering}

\caption[Galaxy-scale obscuration of AGN]{\label{fig:gasobscprofilez_obs}Galaxy-scale obscuration of AGN.
At various redshift intervals we show the fraction of AGN (y-axis)
that is covered by a given column density $\NH$ (x-axis). Red lines
indicate our derivations with uncertainties shown in red shades. Galaxy-scale
obscuration is negligible for column densities of $10^{24}{\rm cm}^{-2}$,
but a important contributor to the AGN fraction at $\NH\approx10^{22-23}\,\text{cm}^{-2}$.
Data points represent fractions from AGN surveys: Bright AGN (large
triangles) show lower obscurations than faint AGN (squares). The data
points are black, yellow, green small symbols \citep[respectively]{Ueda2014,Aird2015,Ricci2016}
and black and coloured large triangles \citep[respectively]{Burlon2011,Buchner2015}.
Since AGN also have a nuclear obscurer, these should be regarded as
upper limits for galaxy gas obscuration (red). Arguable consistent
with all data points is a normal distribution around $10^{22}{\rm cm}^{-2}$
(dashed grey curve), which is kept constant across panels.}
\end{figure*}


The obscured fraction of the AGN population from putting together
the observed relationships are shown in Figure \ref{fig:gasobscprofilez_obs}.
Each panel represents a specific redshift. Red lines show our fraction
of obscured AGN (y-axis) for a given column density $\NH$ (x-axis),
assuming various SMF and SARDs, with shades showing the uncertainties
stemming from the obscuration relation.

Firstly, galaxy-scale gas does not provide Compton-thick column densities
($\NH>\sigma_{T}^{-1}=1.5\times10^{24}{\rm cm}^{-2}$). This is because
massive galaxies which reach those densities are rare and therefore
represent a negligible fraction of the AGN population. In contrast,
the observed fraction of Compton-thick AGN is $\sim38\%$ (see \citealp{Buchner2015}
and references therein). We can conclude that Compton-thick obscuration
is always associated with the nuclear region. An alternative, theoretical
argument based on the total metal gas mass present in galaxies is
laid out in Appendix \ref{sub:semi-analytic} and arrives at the same
conclusion.

We therefore focus on the Compton-thin obscurer and compare our obscuration
results to the fraction of Compton-thin AGN with $\NH>10^{22}\text{cm}^{-2}$,
a common definition of ``obscured'' AGN. This step implicitly assumes
that the Compton-thick nuclear obscurer is randomly oriented with
respect to the galaxy, in accordance with chaotic accretion \citep{King2006}.

We now compare to measurements of the obscured fraction of Compton-thin
AGN from surveys. When these fractions are treated as covering fractions,
they contain both galaxy-scale obscuration and nuclear obscuration.
Therefore the data points should always be understood as upper limits
to the galaxy-scale gas obscuration. Figure~\ref{fig:gasobscprofilez_obs}
shows results from surveys at the peak of the accretion history ($z=1-3$
panels) from \citet{Ueda2014} and \citet{Buchner2015}, as well as
surveys which include the local Universe \citep{Burlon2011,Ueda2014,Aird2015,Ricci2016}.
Two cases are important: (1) the fraction of obscured AGN at the bright
end ($\LX\geq10^{45}\text{erg/s}$), where the lowest obscured fractions
are observed ($f_{\text{bright}}\sim40\%$, shown as triangles), and
(2) at the faint end ($\LX\sim10^{43}\text{erg/s}$), where the highest
obscured fractions are observed ($f_{\text{faint}}\sim75\%$, shown
as squares). Our results are, as discussed in Section \ref{sec:Methodology},
luminosity-independent. When comparing the triangle data points at
$\NH=10^{22}\text{cm}^{-2}$ in Figure~\ref{fig:gasobscprofilez_obs}
representing the bright end to our results (red), we find broad agreement
within two-sigma uncertainties (red shades). However, due to the large
uncertainties also the faint end may be in agreement. Therefore we
can not distinguish whether galaxy-scale obscuration decreases toward
the bright end, or increases toward the faint end. Nevertheless we
can conclude that galaxy-scale obscuration is an important, if not
dominant, AGN obscurer at $\NH=10^{22-23}\text{cm}^{-2}$. At higher
column densities, e.g. $\NH=10^{23.5}\text{cm}^{-2}$, the observed
obscured fraction is clearly higher than our results from galaxy-scale
gas. Therefore a nuclear obscurer is necessary to explain the elevated
data points. The shape of the distribution is driven by the Gaussian
distribution of the $\NH-M_{\star}$ relation, a shape assumed in
Paper I to fit the dispersion. Adopting a different distribution,
would flatten the tails and permit unobscured LOS.

Note that our results are meaningful for the AGN population -- the
obscuration of individual host galaxies is stellar-mass dependent
with substantial variations between individual galaxies (see Equation
\ref{eq:NHM-rel}).




\section{Discussion}

\label{sec:Discussion}

\label{sub:Galaxy-scale-obscuration-for}Our main result is that the
host galaxy gas provides a luminosity-independent obscurer, for which
we compute covering fractions. This obscurer does not provide Compton-thick
columns, but large covering fractions ($40-90\%$) at $\NH>10^{22}\,{\rm cm}^{-2}$.
These high covering fractions suggest that a substantial part of the
type1/type2 dichotomy is due to galaxy-scale gas. This is consistent
with the finding of \citet{Maiolino1995} that nearby type2 AGN are
often edge-on galaxies. We note that our results tend to produce high
covering fractions, which are only consistent with the measurements
due to large uncertainties. We speculate that the use of GRB host
galaxies in Paper~I may slightly bias our results when applied to
AGN host galaxies. GRB host galaxies are currently star forming and
thus may have slightly more gas than the average AGN galaxy (see caveats
discussed in Section \ref{sec:Methodology}). A correction in the
column density by a factor of 2 would agree well with data points
at all redshifts. This is demonstrated by the dashed line in Figure~\ref{fig:gasobscprofilez_obs},
which shows a normal distribution around $\log\NH/{\rm cm}^{-2}=10^{22}$
with width $\text{0.5}$. Nevertheless, our results are consistent
with the current AGN surveys.

Our results give, for the first time, constraints on the galaxy-scale
obscurer alone. We use this in the following sections to disentangle
the AGN nuclear obscurer from the galaxy-scale obscurer (Section \ref{sub:Ldep}),
and to describe their behaviour as a function of accretion luminosity,
host galaxy stellar mass and redshift (Section \ref{sub:PuffedTorus}).
Finally physical effects leading to this behaviour are discussed in
Section \ref{sub:Physical-effects}.


\subsection{Luminosity-dependence of AGN obscuration}

\label{sub:Ldep}We can now discuss the luminosity-dependence of the
obscurer. X-ray surveys consistently find a strong decline towards
high luminosities \citep{Ueda2003,Hasinger2005,LaFranca2005,Ebrero2009,Ueda2014,Buchner2015,Aird2015}
of the obscured fraction $\NH>10^{22}\,{\rm cm}^{-2}$ in the Compton-thin
AGN (CTNAGN) population. In Figure~\ref{fig:L-dep-sketch} we have
sketched this decline from $\sim70\%$ to a persistent $\sim40\%$.
According to our results, this can be interpreted as two scales contributing
to the obscuration: a $40\%$ galaxy-scale obscurer, and a luminosity-dependent
nuclear obscurer. The former is constrained from GRB observations
(Paper~I), re-weighted to the mass distribution of AGN (results of
this paper). The latter is the remainder to fit the observations.
According to this picture, the nuclear obscurer completely disappears
toward high luminosities at $L(2-10\,{\rm keV})\approx10^{44.5}{\rm erg/s}$.
This luminosity is however redshift-dependent, with higher redshifts
having higher cut-off luminosities. This has been found in many works
\citep{Ueda2003,Ebrero2009,LaFranca2005,Ueda2014,Aird2015} by fitting
a empirical, parametric model to the relative number density derived
from AGN surveys. \citet{Buchner2015} derived the same result using
a non-parametric method, indicating that this is indeed a robust result.
In \citet{Buchner2015}, the implications for obscurer models were
discussed in their Section 5.3. They concluded that an Eddington-limited
blow-out of the obscurer could explain the luminosity dependence.
As the luminosity dependence is observed to evolve, a black hole mass
dependence needed to be invoked. Under the assumption of black hole
mass downsizing, i.e., that the average accreting black hole is more
massive at high redshift than at low redshift, the turn-over luminosity
decreases over cosmic time. Black hole mass downsizing has evidence
from optical observations of the black hole mass function evolution
\citep{Schulze2010,Kelly2013a,Schulze2015} and semi-analytic models
which reproduce the evolution of the AGN luminosity function \citep{Fanidakis2012,Enoki2014,Hirschmann2014}.
Recently, \citet{Oh2015} found evidence that the type-1 fraction
in a SDSS selected sample is both luminosity and black hole mass dependent.

\begin{figure}
\begin{centering}
\includegraphics[width=0.8\columnwidth]{/mnt/data/daten/PostDoc/research/agn/sim/compiled/bh/gasobscfracLsingle}
\par\end{centering}

\caption{\label{fig:L-dep-sketch}The observed luminosity-dependence of the
obscured fraction. With the galaxy-scale gas obscuration constrained
and luminosity-independent (red, Section~\ref{sec:Results}), we
separate out the nuclear, variable obscurer (cyan) and create a luminosity-dependent
model (Section~\ref{sub:PuffedTorus}).}
\end{figure}



\subsection{PuffedTorus - a unified obscurer model}

\label{sub:PuffedTorus}Since neither semi-analytic nor hydro-dynamic
cosmological simulations can resolve the nuclear obscurer of AGN,
we present a sub-grid model for post-processing. The PuffedTorus model
has few parameters and is constructed so that it reproduces the fraction
of obscured AGN as a function of redshift and luminosity as discussed
above. It can also serve as a summary of observational constraints
when exploring physical models for the obscurer.

We assume that a nuclear Compton-thick obscurer covers a fraction
of the SMBH, $f_{\text{CT}}\sim35\%$. Evidence for this number comes
directly from AGN surveys \citep{Buchner2015}, matching the soft
and hard X-ray luminosity function \citep{Aird2015} and matching
the Compton-thin X-ray luminosity function to the spectrum of the
Cosmic X-ray background \citep{Ueda2014}. Similar fractions are now
also found in local surveys, e.g. \citet{Ricci2016}. 

\begin{table*}
\caption{\label{tab:Parameters}Parameters of the PuffedTorus model. The recommended
values are constrained by local observations as described in Section~\ref{sub:PuffedTorus}.}


\centering{}%
\begin{tabular}{ccc>{\centering}p{7cm}}
Parameter & Symbol & Observed Range & Recommended Value\tabularnewline
\hline 
Compton-thick fraction & $f_{\text{CT}}$ & 20\% - 40\% & 35\% (fixed)\tabularnewline
Host galaxy obscuration & $f_{\text{gal}}$ & 0\% - 60\% & galaxy-dependent, but 40\% on average\tabularnewline
Compton-thin stirred obscuration & $f_{\text{nuc}}$ & 30\% - 50\% & 35\%\tabularnewline
Luminosity-dependence width & $\sigma$ & 0 - 1 & 0.5\tabularnewline
Reference mass & $M_{L43}$ & $10^{7}$- $10^{8}$$M_{\odot}$ & $10^{7.87}M_{\odot}$\tabularnewline
Mass-dependence & $\AP$ & 0 - 2 & 2/3 (consider also 0, 1, 4/3)\tabularnewline
Intrinsic luminosity ($2-10\,{\rm keV}$) & $L$ & (dynamic) & Determined from accretion rate using bolometric corrections of \citet{Marconi2004}\tabularnewline
\end{tabular}
\end{table*}
\begin{figure}
\begin{centering}
\includegraphics[width=0.99\columnwidth]{/mnt/data/daten/PostDoc/research/agn/lf/torus/mockmodel6_ML_plane}
\par\end{centering}

\caption{\label{fig:Covering-fraction}Covering fractions as a function of
accretion luminosity and mass of the proposed PuffedTorus model. Dashed
lines indicate constant Eddington accretion rates. At intermediate
accretion rates, the torus is puffed up and covers large fractions
of the black hole.}
\end{figure}
\begin{figure*}
\begin{centering}
\includegraphics[width=1\textwidth]{agnpaper-figure0}
\par\end{centering}

\caption{\label{fig:Cartoon}Cartoon of the three known obscurer components:
Galaxy-scale obscuration (light blue), luminosity-dependent nuclear
obscuration (blue) and Compton-thick nuclear obscuration (red). In
this illustration, the disappearing Compton-thin obscurer is provided
by a clumpy accretion disk wind, while the Compton-thick obscurer
is a clumpy donut-like structure. Both are embedded in the host galaxy
under random orientation. This is one scenario appropriate for the
proposed PuffedTorus model.}
\end{figure*}
We propose that the remaining Compton-thin sky is obscured by galaxy-scale
gas as well as a nuclear Compton-thin obscurer according to the formula:

\begin{eqnarray}
f_{\text{cov}} & = & f_{\text{gal}}+f_{\text{nuc}}\cdot\exp\left\{ -\frac{1}{2\sigma^{2}}\left(\log\frac{L}{L_{\text{peak}}}\right)^{2}\right\} \label{eq:lumdep}
\end{eqnarray}
The luminosity-independent obscuration, $f_{\text{gal}}$ is on average
$\sim40\%$. In hydro-dynamic simulations it can be derived for individual
galaxies through ray-tracing (see Section~\ref{sec:Cosmological-simulations}),
or otherwise calculated from Equation~\ref{eq:NHM-rel}. The obscuration
of Equation \ref{eq:lumdep} reaches a average maximum of $75\%$
at luminosity $L_{\text{peak}}$ (see Figure~\ref{fig:L-dep-sketch}),
and we therefore suggest $f_{\text{nuc}}=35\%$. We chose a Gaussian
form which requires fewer parameters than a linear decline \citep[e.g.][]{Ueda2014}.
The obscured fraction declines towards both bright and faint ends
to $f_{\text{gal}}$ with characteristic width of the transition defined
by $\sigma$. Evidence for the low-luminosity decline was found in
surveys of the local Universe \citep{Burlon2011,Brightman2011b} as
well as at high redshifts \citep{Buchner2015}.

As motivated in the above Section~\ref{sub:Galaxy-scale-obscuration-for},
the peak luminosity $L_{\text{peak}}$ is in turn a function of mass:

\begin{align*}
\frac{L_{\text{peak}}}{10^{43}\text{erg/s}} & =\left(\frac{M_{\text{BH}}}{M_{L43}}\right)^{\AP}
\end{align*}
Here, at $M_{L43}$ the distribution peaks at $L(2-10\,{\rm keV})=10^{43}\text{erg/s}$,
which is $L_{\text{bol}}=10^{10.7}L_{\bigodot}$ using the conversion
of \citep{Marconi2004}. The mass-dependence is defined by the $\AP$
parameter: at $\AP=0$, the PuffedTorus is mass-independent, corresponding
to a strict luminosity-dependent unified model, while at $\AP=1$
it is only Eddington-rate dependent. The parameters $\AP$, $M_{L43}$
and $\sigma$ are not known apriori. For mass-scaling, $\AP=2/3$
is suggested from the theoretical works on the obscurer by \citet{Elitzur2016}
and \citet{Wada2015}, however a wide range (e.g. $\AP=0-2$) may
be considered \citep[see also][]{Hoenig2007}. We derive fiducial
values for the other two parameters using the Swift/BAT survey of
local AGN. That survey reported a mean mass of $\log M_{\text{BH}}/M_{\odot}=7.87$
with a standard deviation of $0.66\,{\rm dex}$ and a skew towards
low masses \citep{WinterBAT2009}. Adopting an appropriate skewed
normal distribution with skew parameter $-10$ (tail to low masses)
around $M_{L43}=10^{7.87}M_{\odot}$, we find that $\sigma\approx0.5$
approximately reproduces the width of the obscured fraction function
reported in \citet{Burlon2011}, and peaks at $\LX=10^{43}\text{erg/s}$.
Table \ref{tab:Parameters} lists the parameters of the PuffedTorus
model, with recommended typical values. We emphasise that the observations
pertaining the redshift evolution have not been used when constructing
our model. Cosmological simulations using the PuffedTorus model can
thus compare against those \citep[e.g.][]{Buchner2015}. 

The ratio of dust re-radiated infrared luminosity to bolometric, illuminating
luminosity has been used to measure the irradiated area (obscurer
covering factor) of individual AGN. Surveys employing this method
\citep[e.g.][]{Maiolino2007,Lusso2013} typically find, once corrected
for anisotropic illumination and emission \citep[see their Figure 13]{Stalevski2016},
fractions between $40\%$ and $75\%$ (luminosity-dependent). Those
measurements would include emission from the Compton-thick obscurer
and the nuclear, Compton-thin obscurer. With our assumed fiducial
values we obtain $35\%$ ($f_{\text{CT}}$) to $75\%$ ($f_{\text{CT}}+f_{\text{nuc}}$)
and are therefore also in agreement with those observations. Infrared
studies remain however difficult to use as a constraint as the entire
infrared SED has to be constrained for each object \citep{Netzer2016}
and covering factors have to be corrected based on uncertain model
geometries \citep{Stalevski2016}. 

Figure~\ref{fig:Covering-fraction} illustrates the behaviour of
the PuffedTorus model. The obscured fraction undergoes a luminosity
and mass-dependent peak, where the Compton-thin medium is extended
and occupies a substantial fraction of the sky. Due to the host galaxy,
a constant fraction is present at all luminosities and masses. The
Compton-thick fraction here has been assumed to be constant, albeit
we note that \citet{Ricci2016} claims a luminosity-dependence of
the Compton-thick fraction.


\subsection{Physical obscurer processes}

\label{sub:Physical-effects}Physical processes giving rise to the
luminosity and mass-dependent behaviour can not be discussed rigorously
within the scope of this paper. However, we point out a few key results
of recent theoretical works. Otherwise we refer to the review of \citet{Hoenig2013}
which discusses current torus modelling approaches. 

We take note of the analytic wind model formalism described in \citet{Elitzur2016}
and of the radiation-driven fountain model by \citet{Wada2015}. Both
models produce a vertically extended obscurer structure as a function
of luminosity and mass. At low luminosities, radiation is not sufficient
to puff up the obscurer. In the hydro-dynamical simulations of \citet{Wada2012}
very high accretion luminosities are associated with strong outflows,
which suppress the vertical extent of the obscuring structure as they
occupy larger angles. The cartoon of Figure~\ref{fig:Cartoon} illustrates
such a possible wind scenario for the three, distinct obscuring components,
with approximately the correct opening angles. For visualisation we
have smooth gas distributions, while in reality the obscurer is thought
to be clumpy \citep[see e.g.,][and references therein]{Markowitz2014}.
To date, models of the nuclear obscurer largely lack observational
constraints. Our PuffedTorus model summarises observational boundary
constraints of covering fractions and column densities for how a physical
``torus'' model should behave.

A further, commonly overlooked aspect is the evolution of AGN. Faint
AGN are much more common than luminous AGN \citep[e.g.][]{Barger2005,Ueda2003,Aird2010},
a fact that has to be explained with the triggering and light-curve
of accretion events. Galaxy-galaxy mergers are today thought to be
the main trigger of luminous AGN activity, because SMBH need to accrete
substantial fractions of their host galaxy gas (as shown by scaling
relations), within durations comparable to dynamical timescales of
galaxy centres \citep{Somerville2008}. In the timeline proposed by
Hopkins et al, the luminous phase occurs in relatively late merger
phases. Luminous AGN stop further infall by radiation pressure and
quickly reduce their column densities \citep{Hopkins2006OriginModel}.
In contrast, the faint AGN population is suggested to be dominated
by periods before and after peak accretion \citep{Hopkins2005LFinterpretation}.
Early merger phases may have enhanced obscuration, both galaxy-scale
(Compton-thin) and nuclear (Compton-thin and Compton-thick) \citep{Hopkins2005,Hopkins2006OriginModel}.
Additionally, a substantial part of the faint AGN population is probably
associated to secular triggering mechanisms \citep[e.g.][]{Treister2012}.
One may therefore argue for a evolutionary AGN life {[}illustrative
lifetimes in brackets{]} consisting of 
\begin{itemize}
\item no accretion, and therefore no AGN detection {[}89\%{]},
\item nearby gas leading to obscuration and triggering of a faint AGN {[}10\%{]},
either in the onset of a merger or due to secular events,
\item major merger triggering a bright AGN, which immediately clears the
vertically extended obscurer but shines for a period of time {[}1\%{]}
before fading.
\end{itemize}
Such a duty cycle would give rise to the observed obscured fractions
(Figure \ref{fig:L-dep-sketch}) while also respecting the luminosity
function of AGN. To summarise, luminous AGN and faint AGN may live
in different environments with different gas reservoirs feeding them;
therefore unifying the obscuration properties of a luminosity-dependent
torus may not be appropriate. 

\pagebreak{}


\section{Cosmological hydro-dynamic simulations}

\label{sec:Cosmological-simulations}We now assess the gas content
in simulated galaxies. Modern cosmological hydro-dynamic simulations
self-consistently evolve galaxies and their processes (star formation,
gas accretion, supernova and AGN feedback, etc.) in the context of
well-constrained $\Lambda$CDM cosmologies. These simulations are
constrained in their initial conditions to the baryon density available
in the early Universe and are tuned to reproduce the local stellar
mass functions. We consider two state-of-the-art cosmological hydro-dynamic
simulations which also produce realistic galaxy morphologies. These
simulations allow us to look at the spatial distribution of gas inside
galaxies.


\subsection{Simulation sets}

The Evolution and Assembly of Galaxies and their Environment (EAGLE)
simulation \citep{Schaye2015,Crain2015} reproduces many observed
quantities; it reproduces very well the stellar mass function \citep{Furlong2015a}
and size distribution \citep{Furlong2015} of galaxies as a function
of cosmic time, being tuned to reproduce these at $z=0$. Further
relevant for this work it also produces galaxies with realistic galaxy
morphologies \citep{Schaye2015} and gas contents consistent with
observations of CO and HI \citep{Bahe2015} as well as H$_{2}$ \citep{Lagos2015}.
This encourages us to look inside simulated galaxies and assess the
obscuration provided by them. EAGLE includes black hole particles,
which are seeded into dark matter halos exceeding masses of $10^{10}M_{\odot}$.
These black holes are kept near the galaxy centre and may accrete
when gas is nearby, in turn activating a feedback mechanism by heating
\citep{Schaye2015}. The strength of EAGLE lies in its minimalistic
subgrid recipes and the systematic exploration of alterations: Besides
the reference model (\texttt{L0100N1504\_REFERENCE}), a series of
simulations with parameter variations have been run to explore the
impact of various choices in the subgrid implementations, including
the style and strength of supernova feedback, AGN heating and criteria
for when stars are formed \citep{Crain2015}. 

We also consider Illustris \citep{Vogelsberger2014,Vogelsberger2014a},
another hydro-dynamic cosmological simulation. This simulation also
reproduces many observed quantities; most relevant for this work is
that it reproduces the morphology of galaxies, the gas content from
CO observations \citep{Vogelsberger2014,Genel2014}. The Illustris
sub-grid models were chosen in consideration of the stellar mass function,
star formation history and mass-metallicity relation. However, the
weak tuning gave mediocre agreement with regards to the galaxy stellar
mass function \citep{Schaye2015} and size distribution \Citep{Furlong2015}.
On the positive side, the Illustris simulation is based on the AREPO
hydrodynamics code which has been shown to reproduce galaxy features
well \citep[e.g.][]{Vogelsberger2012,Nelson2013}. The gas particle
resolution in \emph{Illustris} is adaptive, with some cells being
as small as $48\,{\rm pc}$ in the highest resolution simulation (\emph{Illustris-1})
used here, indicating that modern cosmological simulation indeed resolve
galaxies into small sub-structures. Illustris also includes black
holes, which are created and kept in the gravitational potential minimum
of galaxies inside halos of mass $M_{\text{DM}}>7.1\times10^{10}M_{\odot}$
\citep{Sijacki2015}. 


\subsection{Methodology}

We first investigate the gas distribution in the reference simulations.
We focus on the metal component of gas as O and Fe are, for the relevant
obscuring columns and redshifts, the most important elements for photo-electric
absorption of X-rays. In galaxy evolution models, the massive end
of the existing stellar population expels metals into the galaxy.
The metal gas produced per stellar mass is determined by the chosen
IMF and the metal yield, with the latter tuned to reproduce the stellar
mass function \citep[e.g.][]{Lu2015}. The total metal gas mass residing
in galaxies further depends on the chosen feedback models which can
expel gas out of the galaxy. Typically the metal gas mass inside galaxies
follows a $M_{Z}:M_{\star}$ ratio of $1:30$ to $1:100$ relation
in semi-analytic models at $z=0-3$ \citep[e.g.][]{Croton2006,Croton2016};
Plots of these models can be found in Appendix~\ref{sub:semi-analytic}.
The crucial remaining question surrounds the arrangement of that gas
inside galaxies, as the concentration of gas defines its column density.

\begin{figure*}
\begin{centering}
\includegraphics[width=0.8\paperwidth]{/mnt/data/daten/PostDoc/research/agn/sim/compiled/bh/gasobscprofilez_sim}
\par\end{centering}

\caption[Galaxy-scale obscuration of AGN]{\label{fig:gasobscprofilez_sim}Hydro-dynamic cosmological simulations
results for the galaxy-scale gas obscuration of AGN. At various redshift
intervals (panels) we show the fraction of AGN (y-axis) that is covered
by a given column density $\NH$ (x-axis). Curves indicate results
from ray-tracing the metal gas. The EAGLE reference simulation (thick
line) produces thicker column densities than Illustris (thin line).
Data points are the same as in Figure~\ref{fig:gasobscprofilez_obs}.
These show fractions from surveys of bright/faint AGN in triangle/square
symbols respectively. Since AGN also have a nuclear obscurer, these
should be interpreted as upper limits for galaxy gas obscuration.
The dashed grey curve (same as in Figure~\ref{fig:gasobscprofilez_obs})
is kept constant across panels for reference.}
\end{figure*}


We apply ray tracing, starting from the most massive black hole particle
of each simulated galaxy (subhalo). From that position, we radiate
in random sight-lines all metal gas bound to the subhalo. Along the
ray we assign each part the density from the nearest gas particle
and finally sum to a total metal column density. This Voronoi tessellation
ray tracer can be found at \href{https://github.com/JohannesBuchner/LightRayRider}{https://github.com/JohannesBuchner/LightRayRider};
Catalogues of the obscuration of all considered simulated galaxies
are available from the first author on request. We then compute an
equivalent hydrogen column density distribution by adopting \citet{Wilms2000}
local inter stellar medium (ISM) abundances. This mimics how $\NH$
is derived in X-ray observations. For completeness, Appendix~\ref{sub:Metallicity-sim}
investigates the hydrogen gas and the metallicity of LOS in the simulation.
We adopt $h=0.7$ and work in physical units at redshift slices $z=0,\,1,\,2$
and $3$. We investigate all galaxy subhaloes with black holes. For
each we randomly assign a luminosity according to the SARD of \citet{Aird2012}
and use only those with $L(2-10\,{\rm keV})>10^{42}\text{erg/s}$.
The last step is repeated 400 times to increase the sample size. We
therefore do not rely on the instantaneous accretion rates provided
by the simulations \citep[see][ for Illustris and EAGLE respectively]{Sijacki2015,Rosas-Guevara2016}.
The effect of adopting these as the selection criterion is discussed
later. With the column density distribution for each AGN available,
we compute the obscured fraction as a function of column density $\NH$
of the simulated population.


\subsection{Results}

\label{sub:Results}We present the fraction of AGN showing column
densities larger than a given $\NH$ value in Figure \ref{fig:gasobscprofilez_sim}.
The plot is made in the same fashion as the previous observational
Figure~\ref{fig:gasobscprofilez_obs} and compares against the same
observations. In the upper right, $z=1$ panel of Figure \ref{fig:gasobscprofilez_sim},
we find that both the EAGLE reference and Illustris simulations produce
a negligible number of Compton-thick AGN. This is consistent with
the assumption that Compton-thick obscuration is associated with a
nuclear obscuration in the vicinity of the accretion disk. We then
compare to observations. Downwards-pointing triangles indicate constraints
for the obscured fraction of luminous, Compton-thin AGN. Arbitrary
additional nuclear obscuration may be included in them, therefore
they should be interpreted as upper limits to the galaxy-scale obscuration.
We find that the Illustris simulation fulfils these constraints, as
it produces very low covering fractions at all obscuring columns.
In contrast, the EAGLE reference simulation produces an excess of
obscured AGN at $\NH=10^{22}\text{cm}^{-2}$. This is in violation
of observations even when the higher data point from low-luminosity
AGN is considered. At higher column densities, the large-scale galaxy
gas of the EAGLE reference simulation produces covering fractions
consistent with the observations, with no need for a nuclear obscurer
up to $\NH=10^{23.5}\text{cm}^{-2}$. In contrast, Illustris galaxies
do not provide column densities of $\NH>10^{23}\text{cm}^{-2}$ and
thus require a nuclear obscurer to explain the observations. The same
trends are seen at higher redshifts in the panels of Figure \ref{fig:gasobscprofilez_sim}.
At redshift $z=0$, both EAGLE and Illustris are consistent with the
data points.


\subsection{Discussion}

\begin{figure}
\begin{centering}
\includegraphics[width=1\columnwidth]{/mnt/data/daten/PostDoc/research/agn/sim/compiled/bh/gasobscprofilesingle}
\par\end{centering}

\caption[Obscured fractions of various simulations]{\label{fig:gasobscprofile-many}Obscured fractions from various simulations.
We compare the fraction of obscured, Compton-thin AGN at $z=1$ from
observations (first two data points) to our observational model (Section
\ref{sec:Methodology}) and various cosmological simulations.}
\end{figure}


We discuss three aspects which affect the results: (1) different sub-grid
physics, most notably stronger feedback mechanisms, (2) differences
between active and passive galaxies, (3) unresolved substructure of
the gas.

The strength of EAGLE is that we can explore how variations of the
physics affect the results. The diversity of those predictions is
bounded by dotted lines in Figure~\ref{fig:gasobscprofilez_sim}.
This indicates that at least some models reproduce fractions in agreement
with observation. However it also shows that the simulations are limited
in their predictive power, as arbitrary fractions can be produced
depending on the input physics. However, we can use observed obscured
AGN fractions to exclude simulations which overproduce them, as these
observations are apparently quite sensitive to the assumed physics.
We focus on the fraction of AGN with $\NH>10^{22}\text{cm}^{-2}$
at redshift $z=1$, and compare to EAGLE physics variations in Figure~\ref{fig:gasobscprofile-many}.
Observations find a fraction around $\sim30-50\%$. We consider any
simulation with fractions above $2/3$ (right dotted line) as ruled
out by observations. This nicely separates the EAGLE physics variations
into two groups, one close to the observational constraints, and one
significantly over-predicting the fraction of obscured AGN. The EAGLE
reference simulation in a large cosmological volume (\texttt{L0100N1504\_REFERENCE})
belongs to the latter group, as well as the simulation run in medium-size
volumes (\texttt{L0025N0752\_REFERENCE}, \texttt{L0025N0752\_RECALIBRATED}).
We now investigate which changes make the simulation agree better.
\citet{Crain2015} presents these variations in detail.

Star formation-related feedback (supernovae, stellar winds, radiation
pressure, cosmic rays) was altered in the WeakFB and StrongFB models.
The efficiency threshold was modified by factors of $0.5$ and $2$
respectively, relative to the reference model. Here we find, surprisingly,
that both models produce lower obscured fractions than the reference
model. Strong feedback leads to underproduction of massive galaxies
at $M_{\star}>10^{10.5}M_{\odot}$ \citep[their Figure 10]{Crain2015},
thereby biasing the galaxy population to small gas masses. It is less
clear why the WeakFB produces a small obscured fraction. Presumably
the feedback is not sufficient to vertically puff up galaxies, thereby
reducing covering fractions. However, these variations can be excluded
based on their mismatch with other observations, e.g. the SMF \citep[their Figure 10]{Crain2015}.

Next we discuss the effects of feedback from accreting SMBHs. In the
EAGLE simulation, there are three parameters which affect the triggering,
efficiency and impact of AGN feedback respectively. The equation of
state of the ISM can be modified from its reference value (4/3) to
$1$ (\texttt{eos1}). The increased sound speed then increases the
accretion onto black holes in the simulation, which increases AGN
feedback. Once near the black hole, matter is placed into the black
hole with a Bondi accretion formula modified by a viscosity parameter.
Increasing the viscosity (\texttt{ViscHi}) allows gas to loose angular
momentum and accrete more efficiently. Once accreted, the temperature
of particles near AGN is increased by $\Delta T=10^{8.5}{\rm K}$,
stochastically, bringing the gas into the metal cooling regime. \citet{Schaye2015}
test the impact of increasing this temperature to $\Delta T=10^{9}{\rm K}$
(\texttt{AGNdT9}). Each of these three modification (\texttt{eos1},
\texttt{ViscHi}, \texttt{AGNdT9}) leads to a reduction of the obscured
fraction of AGN (see Figure~\ref{fig:gasobscprofile-many}), while
contrary modifications (\texttt{eos5/3}, \texttt{ViscLo}, \texttt{AGNdT8})
do the opposite. It is worth noting that \citet{Schaye2015} preferred
the \texttt{AGNdT9} variation over the reference simulation because
it better fits soft X-ray observations of cluster gas. We also point
out that Illustris uses relative strong feedback, as it implements
three different AGN feedback schemes (thermal, kinetic and radiation),
whereas EAGLE uses only stochastic heating. 

AGN outflows or radiation pressure may decrease the covering fractions
momentarily. We have so far considered \emph{all} simulated galaxies
and found that they produce high covering fractions. Arguably these
fractions are consistent with the covering fraction of low-luminosity
AGN. Therefore, one could propose that AGN feedback at high luminosities
modifies the galaxy in such way that covering fractions are reduced.
Apriori this proposal appears unlikely, because nuclear gas should
be affected first. Additionally, studies comparing the morphology
of active and passive galaxies have found little evidence that these
are different, by comparing appearances with asymmetry and concentration
measures \citep{Grogin2003,Grogin2005,Pierce2007,Gabor2009,Kocevski2012}
or visual classification \citep{Kocevski2012}. Indeed, our results
remain unchanged if in the EAGLE reference simulation we only consider
simulated galaxies with instantaneous black hole accretion rates corresponding
to $L(2-10\,{\rm keV})>10^{42}{\rm erg/s}$, assuming a radiative
efficiency of 10\% and bolometric corrections of \citet{Marconi2004}.
In fact, since active galaxies are preferentially gas-rich, star-forming
galaxies in that simulation, the average column density is higher
by a factor of 2, which increases the discrepancy.

Clumpy ISM may decrease the covering fractions. For instance the galactic
ISM is structured into parsec-size clumps with filling factors of
$1\%$ \citep{Cox2005}. Such clumps could not be resolved by simulations.
However, as a LOS passes through large distances of the ISM ($1-\text{few}$
kiloparsecs), this clumpiness averages out. Additionally, clumpiness
would effectively only redistribute the obscured fraction to both
lower and higher column densities, potentially violating the constraints
of higher column densities. There are also differences in the hydrodynamics
code schemes and their accuracy. However, these are less important
than the chosen sub-grid models \citep[J. Shaye, priv. comm., see][]{Scannapieco2012,Schaller2015,Cui2016,Sembolini2016}
in the present non-classical SPH simulations.

To summarise, our obscured fraction diagnostic is a highly sensitive
test of feedback recipes. It can easily rule out feedback models already
at early times (e.g. $z=3$) in the simulations, if they produce very
high fractions of obscured AGN.


\section{Conclusions}

\label{sec:Summary}Using only observational relations, we predict
the covering fractions of galaxy-scale gas as relevant for the AGN
population. Our findings can be summarised as follows:
\begin{enumerate}
\item Galaxy-scale gas does not provide Compton-thick lines of sight. 
\item Galaxy-scale gas covers substantial fractions of the SMBH population
at $\NH\approx10^{22-23.5}{\rm cm}^{-2}$, sufficient to explain the
observed luminosity-independent baseline obscuration. 
\end{enumerate}
We therefore conclude that heavily obscured AGN are associated with
nuclear obscuration, and propose the value $\NH=10^{23.5}{\rm cm}^{-2}$
as a demarcation line singling out the nuclear obscurer. 

We subtracted the galaxy-scale obscuration and concluded regarding
the remaining nuclear obscurer, that
\begin{enumerate}
\item a nuclear Compton-thick obscurer with $\sim35\%$ covering is necessary.
\item a nuclear Compton-thin obscurer is necessary for some combinations
of luminosity/black hole mass. 
\end{enumerate}
The result is formalised into a semi-analytic model for cosmological
simulations, called PuffedTorus (Section \ref{sub:PuffedTorus}).
The cartoon of Figure~\ref{fig:Cartoon} illustrates a possible physical
scenario for these three, distinct obscuring components.

We also investigated the inside of simulated galaxies from state-of-the-art
hydro-dynamic, cosmological simulations and apply ray-tracing from
their black holes. Some of these simulations produce obscured fractions
from their metal gas which is consistent with observed populations.
However, the results are highly sensitive to the adopted feedback
models and therefore lack predictive power. We therefore suggest the
Compton-thin obscured AGN fraction as a diagnostic to rule out feedback
models. This diagnostic which can be applied already at early cosmic
times ($z=2-3$).


\section*{Acknowledgements}

JB thanks Antonis Georgakakis and Dave Alexander for insightful conversations.
JB thanks Joop Schaye for detailed comments pertaining the hydrodynamic
simulations work. JB thanks Klaus Dolag, Sergio Contreras and Torsten
Naab for conversations about hydro-dynamic simulations.

\bibliographystyle{mnras}
\bibliography{/mnt/data/daten/PostDoc/literature/agn,/mnt/data/daten/PostDoc/literature/grb,/mnt/data/daten/PostDoc/literature/sim}


\appendix

\begin{figure*}
\begin{centering}
\includegraphics[width=1\textwidth]{../../agn/sim/tao/stellar/sage}
\par\end{centering}

\caption{\label{fig:MZ-sam}The metal gas mass in galaxies according to the
SAGE semi-analytic model. Red indicates the median, while dashed lines
are the $1\sigma$-equivalent quantiles. The ratio of metal gas to
stellar mass lies between $1:30$ and $1:100$.}
\end{figure*}



\section{The gas mass inside galaxy}

\label{sub:semi-analytic}Important constraints on how much gas resides
in galaxies can be drawn from cosmological simulations. Such simulations
evolve the matter density available at the Big Bang into collapsing
bound structures. Semi-analytic models, relying on dark matter haloes
from dark matter N-body simulations have been highly successful in
reproducing many features of galaxies, including the stellar mass
function of galaxies and their colour distribution \citep{Croton2006,Somerville2008,Hirschmann2012,Fanidakis2012}.
As an illustrative case, we consider the model of \citet{Croton2016}.
Figure \ref{fig:MZ-sam} plots the metal gas mass (the X-ray obscurer)
present as a function of galaxy stellar mass. The median (red curve)
falls consistently in the 1:30 to 1:100 range for the ratio of metal
gas mass to stellar mass. With AGN host galaxies primarily drawn around
the $M_{\star}=10^{10-11}M_{\odot}$ regime, the total gas available
to obscure a central point source is about $M_{\text{Z}}\sim10^{9}M_{\odot}$.
The Illustris simulation shows very similar results at $z=1-3$ obeying
the same gas ratios. However at $z=0$, the high-mass end is lacks
gas due to the strong feedback implemented in that simulation.

\begin{figure}
\includegraphics[width=1\columnwidth]{/mnt/data/daten/PostDoc/research/agn/sim/sphere/sphere}

\caption{\label{fig:sim-MZ-obsc}X-ray obscuration a given metal gas mass can
reach. A obscurer of column density $N_{H}$ outside a radius $R$
has a minimum metal gas mass limit $M_{Z}$ (lines). For example,
a $M_{Z}=10^{9}M_{\odot}$ gas mass has to be brought withing 100pc
to completely enshroud a region with Compton-thick column densities.
Keep in mind that such masses are only found in very massive galaxies
($M_{\star}>3\cdot10^{10}M_{\odot}$, see Figure \ref{fig:MZ-sam}).}
\end{figure}
We present a simple calculation to show that Compton-thick column
densities, i.e. $\NH>1.5\times10^{24}\text{cm}^{-2}$) can not be
achieved by accumulating the galaxy gas over several kpc. In X-ray
spectral analysis, the equivalent hydrogen column density $\NH$ is
usually computed assuming solar abundances. To mimic this, we convert
the metal mass in particles to the number of hydrogen atoms assuming
solar mass fractions of the nearby ISM from \citet{Wilms2000}:

\begin{eqnarray*}
n_{\text{H}} & = & \left.\frac{f_{\text{X}}}{f_{\text{Z}}}\right|_{\text{solar}}\times\frac{\rho_{\text{Z}}}{m_{\text{H}}}
\end{eqnarray*}
Inserting numbers including the hydrogen mass $m_{\text{H}}$ we find
\begin{eqnarray*}
n_{\text{H}} & = & 0.737\times10^{22}\text{cm}^{-2}\frac{1}{1\text{kpc}}\times\frac{\rho_{\text{Z}}}{10^{6}M_{\odot}\text{kpc}^{-3}}.
\end{eqnarray*}
For example, a 1kpc ray in a region of metal gas density $10^{6}M_{\odot}/\text{kpc}^{3}$
results in a measured column density of $\NH\approx10^{22}\text{cm}^{-2}$. 

The gas inside a galaxy may be arranged in a multitude of ways to
achieve a covering with column density $\NH$. If we consider only
gas \emph{outside} a certain radius $R$, the most effective obscurer,
i.e. the one with the least mass but complete covering, is an infinitely
thin shell at that radius $R$. Its mass is easily computed as 
\[
M_{\text{H}}(\NH,>R)=4\pi\cdot R^{2}\times\NH\times m_{\text{H}}.
\]
Converting to metals using the factor $\left.\frac{f_{\text{Z}}}{f_{\text{X}}}\right|_{\text{solar}}$and
expressing in conventional units, this limit is
\[
M_{\text{Z}}(\NH,>R)=2.6\times10^{9}M_{\odot}\cdot\frac{\NH}{1.5\times10^{24}\text{cm}^{-2}}\cdot\left(\frac{R}{1\text{kpc}}\right)^{2}.
\]
Therefore, a metal mass larger than $2.6\times10^{9}M_{\odot}$ is
required to create a Compton-thick obscurer outside the central $1\text{kpc}$.
Or equivalently, a metal mass of $2.6\times10^{9}M_{\odot}$ has to
be brought to the central $\text{kpc}$ to act as a Compton-thick
obscurer. Note that this mass limit scales with the covering factor;
for example obscuration of $10\%$ of the sky requires $10\%$ of
the mass. This simple limit is shown for several levels of obscuration
in Figure \ref{fig:sim-MZ-obsc}. \citet{Risaliti1999} already noted
that such large masses at radii further than a few 10 pc are ruled
out in NGC1068 and Circinus because they would gravitationally dominate
the central region.

Combining this simple limit with the masses of Figure \ref{fig:MZ-sam},
we can now conclude that galaxies simply do not have the required
gas to provide Compton-thick obscurers with substantial covering factors
outside the central $1\text{kpc}$. Admittedly, this is a weak constraint.
However the result holds independently of the geometry of the gas,
the type of galaxy and is also applicable to mergers. As an example,
lets consider that a $M_{\star}=10^{9}M_{\odot}$ merges into a $M_{\star}=10^{10}M_{\odot}$
galaxy (minor merger), and lets assume that all of its gas ($M_{Z}\approx10^{7}M_{\odot}$)
is made available. That entire amount of gas must land within $100{\rm pc}$
of the AGN in order to completely enshroud in Compton-thick columns.
More quantitative conclusions depend on the geometrical distribution
of the gas in the galaxy. We analyse the galaxies produced by hydrodynamic
simulations in Section \ref{sec:Cosmological-simulations}.


\section{Metallicity of sightlines to AGN and GRB}

\begin{figure}
\begin{centering}
\includegraphics[width=0.9\columnwidth]{/mnt/data/daten/PostDoc/research/agn/sim/compiled/bh/gascolumnabundance_0}
\par\end{centering}

\caption[AGN LOS column abundances]{\label{fig:Z-sim-1}LOS column abundances to AGN. In the EAGLE reference
simulation, hydrogen and metal gas densities were integrated along
random sightlines from the central black hole to the edge of the host
galaxy; the ratio is presented. Curves present the median, while the
shaded regions represent the $1\sigma$ scatter in the population.
LOS metallicities are in general elevated compared to n the local
ISM and show a stellar-mass dependence.}
\end{figure}


\begin{figure}
\begin{centering}
\includegraphics[width=0.9\columnwidth]{/mnt/data/daten/PostDoc/research/agn/sim/compiled/grb/gascolumnabundance_0}
\par\end{centering}

\caption[AGN LOS column abundances]{\label{fig:Z-sim-2}LOS column abundances to GRBs. Same as in Figure~\ref{fig:Z-sim-1},
but sightlines started from the from the densest regions. }
\end{figure}


\label{sub:Metallicity-sim}This paper focused on metal column densities,
not hydrogen column densities. We now consider the hydrogen column
densities in the reference EAGLE simulation. Our goal is to investigate
whether and how they are different from the usually assumed local
ISM metallicities in X-ray observation of high-redshift sources. \citet{Bahe2015}
investigated already the hydrogen masses and surface densities of
EAGLE galaxies and found good agreement with observations. Here we
investigate the abundance in random sightlines for AGN. The top panel
of Figure~\ref{fig:Z-sim-2} presents the metal abundance $N_{\text{Z}}/\NH$
relative to local ISM abundances of \citet{Wilms2000} as a function
of galaxy mass for AGN. In general, approximately solar abundances
are predicted as the LOS crosses the host galaxy. The abundance increases
over cosmic time as metals build up. Also, there is the usual mass-dependent
increase in metallicity. This effect is less prominent in AGN sightlines
(Figure~\ref{fig:Z-sim-1}) which always end in the metal-rich centre
of galaxies. 

For completeness we also present the expected metallicities for GRB
sightlines in Figure~\ref{fig:Z-sim-2}. For low stellar mass hosts
at high redshift, they are expected to be sub-solar. These hosts dominate
the observed host distribution \citep[see e.g.,][]{Perley2015b},
and therefore sub-solar metallicities are to be expected in LGRB afterglow
spectroscopy. For research on the optical absorption of GRBs by HI
we refer to the simulations of \citet{Pontzen2010}.
\end{document}
